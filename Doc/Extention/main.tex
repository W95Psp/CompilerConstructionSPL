\documentclass{article}

\usepackage{listings}

\author{Lucas Franceschino}
\title{Extension part: closure \& higher order support}

\begin{document}
	\maketitle
	\section{My plans}
		I want my compiler to have higher order function support and closure support. To do so, I have to improve or add the following:
		\subsection{Grammar}
			Change my grammar to allow function definition in function bodies. Then a function would begin by a - possible empty - list of variable or function declarations.

			I would also like to have anonymous functions, so I have to find a syntax for them and add them to the grammar definition.
		\subsection{Blur the line between functions and variables, lambda functions}
			In order to make functions ok to store into "normal" variables, I need to let the user define variables by assignation of "function like" values.

			I also need to let the user use lambda functions. I plan to use the exact same mechanism that I use for normal functions, but with a different syntax and no name (the name will be actually replaced by "lambda\_\{id\}"). 

		\subsection{Type inference}
			Improve my type inference: at the moment, my type inference struggles with some cases of recursive functions.
			At the moment I have a short analysis of function dependencies, but not something strong enough. I plan to rebuild that to make a full dependency graph of all functions and variables and make it able to compute variable liveness.

		\subsection{Closure handling strategy}
			As I said previously, by using a dependency graph, I'll be able to see where a variable is used. If a closure $\lambda$ uses a variable (or function) $x$ and if $\lambda$ can be called outside of its parent scope $p$ (i.e. returned by parent scope), then, we lift the variable $x$ to the heap. It means that if some statements in $p$ need to access or to set $x$, then, they will refer to some place in the heap. Then, if multiple lambda functions using $x$ are returned from $p$ (via a list or a tuple for instance), then $x$ will exist (uniquely) even if the scope $p$ is done.

		\subsection{Closure code generation}
			Let suppose we have the following code:

\begin{lstlisting}
f(a){
	g(c){
		return c * a;
	}
	return g;
}
f(10)(2);
\end{lstlisting}
			Then we can't just return $g$, we need (runtime) information concerning some scope of $f$. Using a dependency analysis, we know $n=1$ variables ($a$) are used in $g$. Then, we make a $n+1$ dimentional tuple in the heap with the hard address of $g$ and the addresses of all different linked variables. There is no need to actually know the number of linked variables of $g$: on calling, we will let $g$ unpacks theses variables addresses.


\end{document}