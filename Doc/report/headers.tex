% Loads of (useful ?) packages
\usepackage[french,english]{babel}
\usepackage[utf8]{inputenc}
\usepackage{amsfonts}
\usepackage{tikz}
\usepackage{tikz-qtree}
\usepackage[shortlabels]{enumitem}
\usepackage[a4paper]{geometry}
\usepackage{graphicx}
\usepackage{color}
\usepackage{url}
\usepackage{mathenv}
\usepackage{xstring}
\usepackage{array}
\usepackage[babel=true]{csquotes}
\usepackage{ifthen}
\usepackage{frcursive}
\usepackage{framed}
\usepackage{mdframed}
\usepackage{ntheorem}
\usepackage{listings}
\usepackage{esvect}
\usepackage{amsmath}
% \usepackage{amsthm}
\geometry{hscale=0.75,vscale=0.75,centering}
% \newtheorem{mdtheorem}{Theorem}

% Define natural-ways to name sets
\newcommand{\reals}{\mathbb{R}} 
\newcommand{\realsSSup}{\mathbb{R^*_+}} 
\newcommand{\rationals}{\mathbb{Q}}
\newcommand{\irrationals}{\mathbb{I}}
\newcommand{\naturals}{\mathbb{N}}
\newcommand{\relatives}{\mathbb{Z}}
\newcommand{\complexes}{\mathbb{C}}
\renewcommand{\implies}{\Rightarrow}
\newcommand{\doubleK}{\mathbb{K}}
\newcommand{\doubleR}{\mathbb{R}}
\newcommand{\doubleC}{\mathbb{C}}
\newcommand{\doubleZ}{\mathbb{Z}}
\newcommand{\equival}{\Leftrightarrow}


\newcommand*\diff{\mathop{}\!\mathrm{d}}



% Changing ugly-standart-sqrt function by a new one (src http://tex.stackexchange.com/questions/214368/sqrt-sign-with-sharp-or-rounded-corner)
\let\oldsqrt\sqrt
\def\sqrt{\mathpalette\DHLhksqrt}
\def\DHLhksqrt#1#2{%
\setbox0=\hbox{$#1\oldsqrt{#2\,}$}\dimen0=\ht0
\advance\dimen0-0.2\ht0
\setbox2=\hbox{\vrule height\ht0 depth -\dimen0}%
{\box0\lower0.4pt\box2}}

% qq is a command to quick quote (qq)
%	 usage: \qq{Bla} produce \og Bla \fg{}
\newcommand{\qq}[1]{\og #1\fg{}}

% pSc
\newcommand{\pSc}[1]{\left\langle #1\right\rangle}

% To write a "quick lim"
\newcommand{\quickLim}[1]{_{\vv{~~#1~~}}~}

% \defF allow you to note functions the following way :
%	f:	E -> F
%		x -> f(x)
% 	 usage: {1: function name, 2:fromSet, 3:toSet, 4:variable, 5:expression}
\newcommand{\defF}[5]{%
	\begin{array}{ccccl}%
		#1 & : & #2 & \to & #3 \\%
		 & & #4 & \mapsto & #1(#4)=#5 \\%
	\end{array}%
}



\newcommand{\vectN}[2][n]{\ensuremath{\left(#2_1, ..., #2_#1\right)}}

\let\oldsup\sup


\newcommand\subrel[2]{\mathrel{\mathop{#2}\limits_{#1}}}

% \renewcommand{\sup}[1]{%
% 	\(\stackrel{\mbox{\small sup}}{\mbox{\tiny #1}}\)
% }


\newcommand{\superpose}[2]{%
	\ensuremath{%
		\subrel{^{\mbox{\tiny $#2$}}}{\mbox{\normalfont #1}}%
	}%
}
\renewcommand{\sup}[1]{%
	\superpose{sup}{#1}%
}

\renewcommand{\inf}[1]{%
	\superpose{inf}{#1}%
}



\newcommand{\norme}[1]{\left\Vert #1\right\Vert}
\newcommand{\extendreals}[1]{\displaystyle{\reals}}


\newcommand{\accol}[1]{\left\{#1\right\}}
\newcommand{\T}{\perp}

\newcommand{\Id}{Id}



\newcommand{\vTwo}[2]{
	\left( \begin{array}{c}
	#1 \\
	#2
	\end{array} \right)
}
\newcommand{\vThree}[3]{
	\left( \begin{array}{c}
	#1 \\
	#2 \\
	#3
	\end{array} \right)
}
\newcommand{\vFour}[4]{
	\left( \begin{array}{c}
	#1 \\
	#2 \\
	#3 \\
	#4
	\end{array} \right)
}
\newcommand{\vFive}[5]{
	\left( \begin{array}{c}
	#1 \\
	#2 \\
	#3 \\
	#4 \\
	#5
	\end{array} \right)
}

\newcommand{\B}[1]{\textbf{#1}}

\newcommand{\vTwoDots}[2]{
	\left( \begin{array}{c}
	#1 \\
	\vdots \\
	#2
	\end{array} \right)
}














\newcommand{\missingStuff}{
	~\\~\\~\\
	\begin{center}
		\textbf{\Large{Manque des choses ICI !!}}
	\end{center}
	~\\~\\~\\
}








\theorembodyfont{\normalfont}



\definecolor{grayLight}{HTML}{ECF0F1}
\definecolor{asbestos}{HTML}{7F8C8D}
\definecolor{peterRiver}{HTML}{3498DB}
\definecolor{alizarin}{HTML}{E74C3C}
\definecolor{carrot}{HTML}{E67E22}
\definecolor{ermerald}{HTML}{2ECC71}
\definecolor{turquoise}{HTML}{1ABC9C}
\definecolor{orangeColor}{HTML}{F39C12}


\theoremstyle{nonumberplain}
\newmdtheoremenv[%
	  backgroundcolor=grayLight,
	  linecolor=peterRiver,
	  linewidth=2pt,
	  topline=false,
	  rightline=false,
	  leftline=true,
	  bottomline=false
  ]{Remarque}{Remarque :}

\theoremstyle{nonumberbreak}
\newmdtheoremenv[%
	  backgroundcolor=grayLight,
	  linecolor=alizarin,
	  linewidth=2pt,
	  topline=false,
	  rightline=false,
	  leftline=true,
	  bottomline=false
  ]{Demonstration}{Démonstration :}

\theoremstyle{nonumberbreak}
\newmdtheoremenv[%
	  backgroundcolor=grayLight,
	  linecolor=ermerald,
	  linewidth=2pt,
	  topline=false,
	  rightline=false,
	  leftline=true,
	  bottomline=false
  ]{Definition}{Définition :}

\theoremstyle{nonumberbreak}
\newmdtheoremenv[%
	  backgroundcolor=grayLight,
	  linecolor=ermerald,
	  linewidth=2pt,
	  topline=false,
	  rightline=false,
	  leftline=true,
	  bottomline=false
  ]{Technique}{Technique:}



\theoremstyle{nonumberbreak}
\newmdtheoremenv[%
	  backgroundcolor=grayLight,
	  linecolor=turquoise,
	  linewidth=2pt,
	  topline=false,
	  rightline=false,
	  leftline=true,
	  bottomline=false
  ]{Proposition}{Proposition :}


\theoremstyle{nonumberbreak}
\newmdtheoremenv[%
	  backgroundcolor=grayLight,
	  linecolor=turquoise,
	  linewidth=2pt,
	  topline=false,
	  rightline=false,
	  leftline=true,
	  bottomline=false
  ]{Rappel}{Rappel :}

\theoremstyle{nonumberbreak}
\newmdtheoremenv[%
	  backgroundcolor=grayLight,
	  linecolor=orangeColor,
	  linewidth=2pt,
	  topline=false,
	  rightline=false,
	  leftline=true,
	  bottomline=false
  ]{Theoreme}{Théorème :}


\newcommand{\insideDemo}{
	\normalfont
	\vspace{3mm}\\
	\textbf{Démonstration :}\\
	\itshape
}

\newcommand{\bItem}{\item[$\bullet$]}

\theoremstyle{nonumberplain}
\newmdtheoremenv[%
	  backgroundcolor=grayLight!30,
	  linecolor=asbestos!80,
	  linewidth=2pt,
	  topline=true,
	  rightline=false,
	  leftline=false,
	  bottomline=true
  ]{apart}{}

\theoremstyle{nonumberplain}
\newmdtheoremenv[%
	  backgroundcolor=grayLight!100,
	  linecolor=asbestos!80,
	  linewidth=1pt,
	  topline=false,
	  rightline=false,
	  leftline=false,
	  bottomline=false
  ]{ExemplesCore}{Exemples :}

\theoremstyle{nonumberplain}
\newmdtheoremenv[%
	  backgroundcolor=grayLight!0,
	  linecolor=asbestos!80,
	  linewidth=2pt,
	  topline=false,
	  rightline=false,
	  leftline=true,
	  bottomline=false
  ]{WhiteBlock}{}

\newenvironment{Exemples}{
	\begin{ExemplesCore}
	~
	\begin{WhiteBlock}
	\normalfont~\\
	\vspace{-6mm}
	\begin{itemize}
	\setlength{\leftmargin}{0pt}
}{
	\end{itemize}
	\end{WhiteBlock}
	\vspace{2mm}
	\end{ExemplesCore}
}

\newenvironment{Exemple}{
	\begin{ExemplesCore}
	~
	\begin{WhiteBlock}
	\normalfont
	\setlength{\leftmargin}{0pt}
}{
	\end{WhiteBlock}
	\vspace{2mm}
	\end{ExemplesCore}
}



\newenvironment{Algo}[3]{
	~\\
	\textbf{Données :} #1
	\\
	\textbf{Résultat :} #2\\
	\textbf{Variables :} #3\\
	\textbf{Début algo}\\
	\begin{WhiteBlock}
}{
	\end{WhiteBlock}
	\vspace{2mm}
	\textbf{FinAlgo}\\
}